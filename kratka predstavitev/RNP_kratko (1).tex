\documentclass[a4paper,12pt]{article}
\usepackage[slovene]{babel}
\usepackage[utf8]{inputenc}
\usepackage[T1]{fontenc}
\usepackage{lmodern}
\usepackage{amsmath}
\usepackage{hyperref} 
\usepackage{graphicx} 
\usepackage{amsfonts}
\usepackage{eurosym}
\usepackage{stackengine}
\usepackage{fancyhdr}
\usepackage{amssymb}
\usepackage{graphicx}
\usepackage{amsbsy}
\usepackage{mathtools}
\usepackage{amsthm}

\usepackage[output-decimal-marker={,}]{siunitx}

\begin{document}
\title{Kratka predstavitev robustnega problema nahrbtnika}
\author{Eva Babnik in Jan Založnik}
\date{November, 2020}
\maketitle


\newpage
\section{Problem nahrbtnika}


Spodobi se, da v uvodu najprej predstaviva klasični problem nahrbtinka (angl. \textit{classical Knapsack Problem})
 v nadaljevanju poročila bova uporabljala kar kratico KP. Na voljo imamo množico n-tih predmetov, ki jo 
označimo z $N = \{1, \dots, n\}$ in nahrbtnik s kapaciteto $c$. Vsak predmet ima pozitivno vrednost 
$p_{j}$ in pozitivno utež $w_{j}$. Problem nahrbtnika nas sprašuje, katero podmnožico predmetov iz N moramo 
položiti v nahrbtnik, da bo skupna vrednost le teh čim večja možna in da ne bo presegla nahrbtnikove kapacitete. 
Torej maksimiziramo skupno vrednost predmetov pri pogoju, da seštevek izbranih uteži ne presega nahrbtinkove zmogljivosti.
Problem lahko predstavimo tudi kot celoštevilski linearni program (CLP):


\begin{equation}
    \tag*{}
     max \sum_{j \in N} p_{j}x_{j}
\end{equation}
\begin{equation}
    \tag*{}
    \sum_{j \in N} w_{j}x_{j} \leq c
\end{equation}
\begin{equation}
    \tag*{}
    x_{j} \in \{0,1\}, j \in N
\end{equation}

\medskip
\noindent Kjer $x_{j}$ zavzame vrednost 1, če j-ti predmet položimo v nahrbtnik,
 sicer zavzame vrednost 0. 

\section{Robustni problem nahrbtnika}
\medskip
Robustni problem nahrbtnika (angl. \textit{Robust Knapsack Problem}) v nadaljevanju 
RKP, je nekakšna nadgranja problema nahrbtnika. Dodaten problem se pojavi pri točnosti 
naših podatkov, in sicer pri utežeh $w_{j}$. Vsak predmet $j$ ima svojo nominalno težo 
$w_{j}$, ki pa je lahko netočna, ampak zanjo vemo, da se nahaja na intervalu 
[$w_{j} - \underline{w}_{j}$, $w_{j} + \overline{w}_{j}$]. Podan imamo tudi celošteviski parameter $\Gamma$,
ki označuje največje možno število predmetov z netočno izmerjeno težo. Pri iskanju 
rešitve problema moramo torej paziti, da bo seštevek vseh novih uteži še vedno manjši od
kapacitete nahrbtnika. Težav z rešitvijo seveda ne bomo imeli, če bodo vse dejanske uteži
nižje oziroma lažje od njene nominalne vrednosti, lahko pa se zgodi tudi najslabši možni 
primer, ko vse uteži zavzamejo zgornjo mejo intervala. Dopustno rešitev, kjer je
$J \subseteq N$ lahko formuliramo na naslednji način: 
\begin{equation}
    \tag*{}
    \sum_{j \in J}w_{j} + \sum_{j \in \hat{J}}\overline{w}_{j} \leq c,\quad \forall \hat{J} \subseteq J \; \text{in}\; |\hat{J}| \leq \Gamma 
\end{equation}


\section{PRISTOP Z DINAMIČNIM PROGRAMIRANJEM}

Naj bo $ \bar{z}(d, s, j) $ najvišji dobiček za dopustno rešitev s skupno težo $d$, kjer so upoštevani 
samo elementi iz množice $ \{1,…,j\} \in N$ in samo $s$ izmed njih doseže zgornjo mejo $\hat{w}_j$. 
Naj bo $z(d, j)$ največji dobiček za dopustno rešitev s skupno težo $d$, kjer so upoštevani samo elementi 
iz množice $\{ 1,…,j \} \in N $ in naj jih samo $\Gamma $ spremeni težo iz nominalne na zgornjo mejo $\hat{w}_j$. 
Torej velja $d = 0, 1, ..., c$; $s = 0, 1, …, \Gamma$ in $ j = 0, 1, …, n$.
Ključna lastnost pravilnosti tega pristopa je predpostavka, da so predmeti razvrščeni po padajoči teži $\hat{w}_j$. 
Problem lahko zapišemo z naslednjimi rekurzivnimi zvezami: 
\begin{equation}
    \tag*{}
    \begin{matrix}
    \bar{z}(d, s, j) = max\{ \bar{z}(d, s, j - 1), \bar{z}(d - \hat{w}_j, s- 1, j - 1) + p_j\} \\
    \text{za} \; \; d = 0,…, c; \; s = 1,…, \Gamma \;\text{in} \;j = 1,…, n. \\ 
    \\
    z(d, j) = max\{z(d, j - 1), z (d - w_j, j - 1) + p_j\} \\
    \text{za} \; \; d = 0,…, c; \; j = \Gamma + 1,…, n.
    \end{matrix}
\end{equation}

\noindent Začetna vrednost je $\bar{z}(d, s, 0) = -\infty$ za $d = 0,...,c$ in s = $0,..., \Gamma$. 
Nato nastavimo $\bar{z}(0, 0, 0) = 0$. Oba zapisa sta med seboj povezana z enakostjo $z(d, \Gamma) = 
\bar{z}(d, \Gamma, \Gamma\}$ za vsak $d$. Optimalno vrednost robustnega problema nahrbtnika dobimo kot:
\begin{equation}
\tag*{}
z^{*} = max \begin{cases}
max\{z(d, n) |\; d = 1, ..., c \\
max\{\bar{z}(d, s, n) |\; d = 1, ..., c; \;s = 1, ..., \Gamma -1\}
\end{cases}
\end{equation}
kjer porabimo tudi celotno kapaciteto nahrbtnika $c^{*} \leq c$.
\par
Algoritem dinamičnega programiranja je sestavljen iz dveh korakov. V prvem koraku dobimo optimalno 
rešitev, ki vsebuje največ $\Gamma$ elementov s povečano težo. V drugem koraku pa nato dobljeno 
rešitev lahko razširimo z dodatnimi elementi z nespremenjeno težo. Algoritem lahko razdelimo na dva 
koraka, ker razvrstitev predmetov po padajoči teži $\bar{w}_j$ zagotavlja, da so v vsaki rešitvi 
elementi z najmanjšimi indeksi (torej tisti, ki so bili v nahrbtnik položeni prej) tisti, ki bodo 
dosegli večjo težo.

\section{NAČRT ZA NADALJEVANJE PROJEKTA}
V nadaljevanju bova v programskem jeziku Python napisala program, ki s pomočjo dinamičnega programiranja 
poišče optimalni profit in pripadajočo optimalno množico predmetov. Nato bova program uporabila pri iskanju 
sestave portelja, kjer bova pokusila najti portfelj z najvišjim donosom v skladu z danimi omejitvami.
\end{document}







